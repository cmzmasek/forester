% SRE, Mon Dec 25 13:00:46 2000

\documentclass[12pt]{report}
\usepackage{fullpage}
\usepackage{times}
\usepackage{epsfig}
%\usepackage{html}               % From the LaTeX2html translator
\usepackage{apalike}
\setcounter{secnumdepth}{2}

\input{macros}

\begin{document}
\bibliographystyle{apalike}

\section{SSI format}

SSI format (Sequence/Subsequence Index format) indexes flatfile
databases by names and/or accessions, enabling fast retrieval.

An SSI index is a binary file that stores sequence names or accessions
as \emph{keys} that it can look up rapidly. It differentiates between
\emph{primary keys} and \emph{secondary keys}.  There is one and only
one primary key per sequence. There can be more than one secondary key
per sequence. Both primary and secondary keys must be unique
identifiers (no two records have the same key). A program (like
HMMER's distributed PVM implementation) that needs to step through
each sequence one at a time can refer to the list of primary keys. A
program solely concerned with flexible sequence retrieval (such as
SQUID's \prog{sfetch}) might consult an SSI index with accessions as
primary keys, and names as secondary keys.

A single SSI file can index multiple sequence data files. This allows
indexing multifile databases (e.g. Genbank flatfile distributions).

The SSI format is relatively simple and may prove useful for other
indexing tasks besides sequence names. HMMER uses SSI format to index
HMM files.

\subsection{Special features of SSI}

SSI superceded 1994's GSI format after human genome sequence files
started exceeding 2 GB filesystem limitations, and after problems in
the HMMER PVM implementation had to be hacked around. SSI has the
following additional features compared to GSI.

\begin{description}
\item[Separate primary key section] 
Primary keys are set apart in a separate section, enabling programs to
step through a guaranteed one-to-one mapping of keys to sequences.  A
secondary key section adds many-to-one mapping of keys to sequences.

\item[Arbitrary filename and key lengths]
File name lengths and key name lengths are effectively unlimited. 

\item[64-bit indexing]
For sequence files exceeding 2GB, on architectures that support 64-bit
filesystems (such as IRIX, Solaris, Tru64 UNIX, FreeBSD...), SSI
supports 64-bit indexing; depending on the system, file sizes may 
theoretically be allowed to range up to millions of terabytes.

\item[Fast subsequence extraction]
SSI can be used to greatly accelerate \emph{subsequence} extraction
from very long sequences (example: human chromosome contigs). The
sequence file must meet certain formatting conditions for this to
work; see below for details.
\end{description}

\subsection{SSI API in SQUID}

\subsubsection{Functions for using a SSI index file:}

\begin{sreapi}
\item[int SSIOpen(char *filename, SSIFILE **ret\_sfp)]

Opens the SSI index file \prog{filename} and returns a \prog{SSIFILE
*} stream through \prog{ret\_sfp}. Returns 0 on success, nonzero on
failure. The caller must eventually close this stream using
\prog{SSIClose()}.  More than one index can be open at once.

Error codes:\\
\begin{tabular}{ll}
\prog{SSI\_ERR\_NOFILE}   & failed to open file; doesn't exist or not readable\\
\prog{SSI\_ERR\_BADMAGIC} & not a SSI file \\
\prog{SSI\_ERR\_NO64BIT}  & it has 64-bit offsets, and we can't support that\\
\prog{SSI\_ERR\_FORMAT}   & file appears to be corrupted\\
\prog{SSI\_ERR\_MALLOC}   & malloc failed \\
\end{tabular}

\item[int SSIGetOffsetByName(SSIFILE *sfp, char *key, int *ret\_fh, SSIOFFSET *ret\_offset)]

Looks up the string \prog{key} in the open index \prog{sfp}.
\prog{key} can be either a primary or secondary key. If \prog{key} is
found, \prog{*ret\_fh} contains a unique handle on the file
that contains {key} (suitable for an \prog{SSIFileInfo()} call, or for
comparison to the handle of the last file that was opened for
retrieval), and \prog{offset} is filled in with the offset in that
file. Returns 0 on success, non-zero on error.

Error codes:\\
\begin{tabular}{ll}
\prog{SSI\_ERR\_NO\_SUCH\_KEY} & key not found \\
\prog{SSI\_ERR\_NODATA}        & fread() failed, file appears to be corrupted\\
\end{tabular}

\item[int SSIGetOffsetByNumber(SSIFILE *sfp, int nkey, int
*ret\_fh, SSIOFFSET *offset)]

Retrieves information for primary key number \prog{nkey}.  \prog{nkey}
ranges from 0..\prog{nprimary-1}. When the key is found,
\prog{*ret\_fh} contains a unique handle on the file that
contains {key} (suitable for an SSIFileInfo() call, or for comparison
to the handle of the last file that was opened for retrieval), and
\prog{offset} is filled in with the offset in that file. Returns 0 on
success, non-zero on error.

Error codes:\\
\begin{tabular}{ll}
\prog{SSI\_ERR\_SEEK\_FAILED}  & failed to reposition in index file\\
\prog{SSI\_ERR\_NO\_SUCH\_KEY} & key not found \\
\prog{SSI\_ERR\_NODATA}        & fread() failed, file appears to be corrupted\\
\end{tabular}

\item[int SSIGetSubseqOffset(SSIFILE *sfp, char *key, int
requested\_start,  int *ret\_fh,
SSIOFFSET *record\_offset, SSIOFFSET *data\_offset, int *ret\_actual\_start)]

Implements \prog{SSI\_FAST\_SUBSEQ}. 

Looks up the string \prog{key} in the open index \prog{sfp}, and
asks for the nearest offset to a subsequence starting at position
\prog{requested\_start} in the sequence (numbering the sequence 1..L).
\prog{key} can be either a primary or secondary key. If \prog{key} is
found, \prog{*ret\_fh} contains a unique handle on the file that
contains {key} (suitable for an SSIFileInfo() call, or for comparison
to the handle of the last file that was opened for retrieval);
\prog{record\_offset} contains the disk offset to the start of the
record; \prog{data\_offset} contains the disk offset either exactly at
the requested residue, or at the start of the line containing the
requested residue; \prog{ret\_actual\_start} contains the coordinate
(1..L) of the first valid residue at or after
\prog{data\_offset}. \prog{ret\_actual\_start} is $\leq$
\prog{requested\_start}.  Returns 0 on success, non-zero on failure.

Error codes:\\
\begin{tabular}{ll}
\prog{SSI\_ERR\_NO\_SUBSEQS}   & this file or key doesn't allow subseq lookup\\
\prog{SSI\_ERR\_NO\_SUCH\_KEY} & key not found \\
\prog{SSI\_ERR\_RANGE}         & the requested\_start is out of bounds\\
\prog{SSI\_ERR\_NODATA}        & fread() failed, file appears to be corrupted\\
\end{tabular}

\item[int SSISetFilePosition(FILE *fp, SSIOFFSET *offset]

Uses \prog{offset} to sets the file position for \prog{fp} (usually an
open sequence file) relative to the start of the file.  Hides the
details of system-dependent shenanigans necessary for file positioning
in large ($>2$ GB) files. Behaves just like \prog{fseek(fp, offset,
SEEK\_SET)} for 32 bit offsets and $<2$ GB files. Returns 0 on
success, nonzero on error.

Error codes:\\
\begin{tabular}{ll}
\prog{SSI\_ERR\_SEEK\_FAILED}  & failed to reposition the file\\
\end{tabular}

\item[int SSIFileInfo(SSIFILE *sfp, int fh, char **ret\_filename, int *ret\_format)]

Given a file handle \prog{fh} in an open index file \prog{sfp},
retrieve file name \prog{ret\_filename} and the file format
\prog{ret\_format}.  \prog{ret\_filename} is a pointer to a string
maintained internally by \prog{sfp}. It should not be free'd;
\prog{SSIClose(sfp)} will take care of it.

Error codes:\\
\begin{tabular}{ll}
\prog{SSI\_ERR\_BADARG}  & no such file n\\
\end{tabular}

\item[void SSIClose(SSIFILE *sfp)]

Close an open \prog{SSIFILE *}.
\end{sreapi}

\subsubsection{Skeleton example code for using a SSI index file:} 

\small\begin{verbatim}
    SSIFILE   *sfp;
    FILE       *fp;	
    int         fh;
    char       *seqfile;
    int         fmt;
    SSIOFFSET  offset;
    
    SSIOpen(``foo.gsi'', &sfp);

    /* Finding an entry by name 
     * (by number, with SSIGetOffsetByNumber(), is analogous)
     */
    SSIGetOffsetByName(sfp, ``important_key'', &fh, &offset);
    SSIGetFileInfo(sfp, fh, &seqfile, &fmt);
    fp = fopen(seqfile, ``r''); /* more usually SeqfileOpen(), using fmt */
    SSIFilePosition(fp, &offset);
         /* read the entry from there, do whatever... */
    free(seqfile);
    fclose(fp);

    SSIClose(sfp);
\end{verbatim}\normalsize

\subsubsection{Functions for creating a SSI index file:}

\begin{sreapi}
\item[int SSIRecommendMode(char *file)]

Examines the file and determines whether it should be indexed with
large file support or not; returns \prog{SSI\_OFFSET\_I32} for most
files, \prog{SSI\_OFFSET\_I64} for large files, or -1 on failure.

\item[SSIINDEX *SSICreateIndex(int mode)]

Creates and initializes a SSI index structure. Sequence file offset
type to be used is specified by \prog{mode}, which may be either
\prog{SSI\_OFFSET\_I32} or \prog{SSI\_OFFSET\_I64}.  Returns a
pointer to the new structure, or NULL on failure. The caller must free
this structure with \prog{SSIFreeIndex()} when done.

\item[int SSIGetFilePosition(FILE *fp, int mode, SSIOFFSET *ret\_offset)]

Fills \prog{ret\_offset} with the current disk offset of \prog{fp},
relative to the start of the file.  {mode} is the type of offset to
use; it must be either \prog{SSI\_OFFSET\_I32} or
\prog{SSI\_OFFSET\_I64}. Returns 0 on success, non-zero on error.

Error codes:\\
\begin{tabular}{ll}
\prog{SSI\_ERR\_NO64BIT}       & 64-bit mode unsupported on this system\\
\prog{SSI\_ERR\_TELL\_FAILED}  & failed to determine position in file\\
\end{tabular}

\item[int SSIAddFileToIndex(SSIINDEX *g, char *filename, int fmt,
int *ret\_fh)]

Adds the sequence file \prog{filename}, which is known to be in format
\prog{fmt}, to the index \prog{g}. Creates and returns a unique
filehandle \prog{ret\_fh} for associating primary keys with this file
using \prog{SSIAddPrimaryKeyToIndex()}. Returns 0 on success, non-zero
on failure.

Error codes:\\
\begin{tabular}{ll}
\prog{SSI\_ERR\_TOOMANY\_FILES}  & exceeded file number limit\\
\prog{SSI\_ERR\_MALLOC}          & a malloc() failed\\
\end{tabular}

\item[int SSISetFileForSubseq(SSIINDEX *g, int fh, int bpl, int rpl)]

Set \prog{SSI\_FAST\_SUBSEQ} for the file indicated by filehandle
\prog{fh} in the index \prog{g}, setting parameters \prog{bpl} and
\prog{rpl} to the values given. \prog{bpl} is the number of bytes per
sequence data line.  \prog{rpl} is the number of residues per sequence
data line.  Caller must be sure that \prog{bpl} and \prog{rpl} do not
change on any line of any sequence record in the file (except for the
last data line of each record). If this is not the case in this file,
\prog{SSI\_FAST\_SUBSEQ} will not work, and this routine should not be
called. Returns 0 on success, non-zero on failure.

\item[int SSIAddPrimaryKeyToIndex(SSIINDEX *g, char *key, int
fh, SSIOFFSET *r\_off, SSIOFFSET *d\_off, int L)]

Puts a primary key \prog{key} in the index \prog{g}, while telling the
index that this primary key is in the file associated with filehandle
\prog{fh} and its record starts at position \prog{r\_off} in that
file.

\prog{d\_off} and \prog{L} are optional; they may be left unset by
passing NULL and 0, respectively. (If one is provided, both must be
provided.)  If they are provided, \prog{d\_off} gives the position of
the first line of sequence data in the record, and \prog{L} gives
the length of the sequence in residues. They are used when
\prog{SSI\_FAST\_SUBSEQ} is set for the sequence file. If
\prog{SSI\_FAST\_SUBSEQ} is not set for the file, \prog{d\_off} and
\prog{L} will be ignored even if they are available, so it doesn't
hurt for the indexing program to provide them; typically it won't know
whether it's safe to set \prog{SSI\_FAST\_SUBSEQ} for the whole file
until the whole file has been read and every key has already been
added to the index.

Through \prog{ret\_kh} it provides a ``handle'' - a unique
identifier for the primary key - that any subsequent calls to
\prog{SSIAddSecondaryKeyToIndex()} will use to associate one or more
secondary keys with this primary key.

Returns 0 on success, non-zero on error.

Error codes:\\
\begin{tabular}{ll}
\prog{SSI\_ERR\_TOOMANY\_KEYS}  & exceeded primary key limit\\
\prog{SSI\_ERR\_TOOMANY\_FILES} & filenum exceeds file limit\\
\prog{SSI\_ERR\_MALLOC}         & a malloc() failed\\
\end{tabular}


\item[int SSIAddSecondaryKeyToIndex(SSIINDEX *g, char *key, char *pkey)]

Puts a secondary key \prog{key} in the index \prog{g}, associating it
with a primary key \prog{pkey} that has already been added to the index
by \prog{SSIAddPrimaryKeyToIndex()}.
Returns 0 on success, non-zero on error.

Error codes:\\
\begin{tabular}{ll}
\prog{SSI\_ERR\_TOOMANY\_KEYS}  & exceeded secondary key limit\\
\prog{SSI\_ERR\_MALLOC}         & a malloc() failed\\
\end{tabular}



\item[int SSIWriteIndex(char *file, SSIINDEX *g)]

Writes complete index \prog{g} in SSI format to a binary file
\prog{file}.  Does all overhead of sorting the primary and secondary
keys, and maintaining the association of secondary keys with primary
keys during and after the sort. Returns 0 on success, non-zero on
error.

Error codes:\\
\begin{tabular}{ll}
\prog{SSI\_ERR\_NOFILE}  & an fopen() failed\\
\prog{SSI\_ERR\_FWRITE}  & an fwrite() failed\\
\prog{SSI\_ERR\_MALLOC}  & a malloc() failed\\
\end{tabular}


\item[void SSIFreeIndex(SSIINDEX *g)]

Free an index structure.
\end{sreapi}
    

\subsubsection{Other SSI functions:}

\begin{sreapi}
\item[char *SSIErrorString(int n)] 

Returns a pointer to an internal string corresponding to error
\prog{n}, a return code from any of the functions in the API that
return non-zero on error.
\end{sreapi}

\subsection{Detailed specification of SSI binary format}

There are four sections to the SSI file:
\begin{sreitems}{\textbf{Secondary keys}}
\item[\textbf{Header}] 
Contains a magic number indicating GSI version number, and 
various information about the number and sizes of things in the index.

\item[\textbf{Files}]
Contains one or more \emph{file records}, one per sequence file that's
indexed. These contain information about the individual files.

\item[\textbf{Primary keys}]
Contains one or more \emph{primary key records}, one per primary key.

\item[\textbf{Secondary keys}]
Contains one or more \emph{secondary key records}, one per secondary key.
\end{sreitems}

All numeric quantities are stored as unsigned integers of known size
in network (bigendian) order, for maximum crossplatform portability of
the index files. \prog{sqd\_uint16}, \prog{sqd\_uint32}, and
\prog{sqd\_uint64} are typically typedef'd as \prog{unsigned short},
\prog{unsigned int}, and \prog{unsigned long long} or \prog{unsigned
long} at SQUID compile-time. Values may need to be cast to signed
quantities, so only half of their dynamic range is valid
(e.g. 0..32,767 for values of type \prog{sqd\_uint16};
0..2,146,483,647 (2 billion) for \prog{sqd\_uint32}; and 0..9.22e18 (9
million trillion) for \prog{sqd\_uint64}).  These typedef's are
handled automatically by the \prog{./configure} script (see
\prog{squidconf.h.in} before configuration, \prog{squidconf.h} after
configuration). If necessary, \prog{./configure}'s guess can be
overridden in \prog{squidconf.h} after configuration.

\subsubsection{Header section}

The header section contains:

\vspace{1em}
\begin{tabular}{llrr}
Variable          & Description                              & Bytes      & Type \\\hline
\prog{magic}      & SSI version magic number.               &  4         & \prog{sqd\_uint32}\\
\prog{flags}      & Optional behavior flags (see below)      &  4         & \prog{sqd\_uint32}\\
\prog{nfiles}     & Number of files in file section.         &  2         & \prog{sqd\_uint16}\\
\prog{nprimary}   & Number of primary keys.                  &  4         & \prog{sqd\_uint32}\\
\prog{nsecondary} & Number of secondary keys.                &  4         & \prog{sqd\_uint32}\\
\prog{flen}       & Length of filenames (incl. '\verb+\0+')         &  4         & \prog{sqd\_uint32}\\
\prog{plen}       & Length of primary key names (incl. '\verb+\0+') &  4         & \prog{sqd\_uint32}\\
\prog{slen}       & Length of sec. key names (incl. '\verb+\0+')    &  4         & \prog{sqd\_uint32}\\
\prog{frecsize}   & \# of bytes in a file record             &  4         & \prog{sqd\_uint32}\\
\prog{precsize}   & \# of bytes in a primary key record      &  4         & \prog{sqd\_uint32}\\
\prog{srecsize}   & \# of bytes in a sec. key record         &  4         & \prog{sqd\_uint32}\\
\prog{foffset}    & disk offset, start of file records       &  \dag         & \dag\\
\prog{poffset}    & disk offset, start of primary key recs   &  \dag         & \dag\\
\prog{soffset}    & disk offset, start of sec. key records   &  \dag         & \dag\\
\end{tabular}
\vspace{1em}

The optional behavior flags are:

\vspace{1em}
\begin{tabular}{lll}
Flag             & Value& Note\\ \hline
\prog{SSI\_USE64}         & $1 \ll 0$ & Large sequence files; all key offsets 64 bit.\\
\prog{SSI\_USE64\_INDEX}  & $1 \ll 1$ & Large index; GSI file itself uses 64-bit offsets.\\\hline
\end{tabular}
\vspace{1em}

The optional behavior flags define whether the SSI file uses large
file (64-bit) offsets.  This issue is discussed in greater detail
below (see ``Large sequence files and large indices''). Briefly: if
\prog{SSI\_USE64} is set, the sequence file is large, and all sequence
file offsets are 64-bit integers.  If \prog{SSI\_USE64\_INDEX} is
set, the index file itself is large, and \prog{foffset},
\prog{poffset}, and \prog{soffset} (that is, all offsets within the
index file itself, indicated as \dag\ in the above table) are 64-bit
integers. \footnote{In the current API it is not expected that
\prog{SSI\_USE64\_INDEX} would ever be set. The current index-writing
API keeps the entire index in RAM (it has to sort the keys), and would
presumably have to be modified or replaced to be able to generate very
large indices.}

The reason to explicitly record various record sizes (\prog{frecsize},
\prog{precsize}, \prog{srecsize}) and index file positions
(\prog{foffset}, \prog{poffset}, \prog{soffset}) is to allow future
extendibility. More fields might be added without breaking older SSI
parsers.  The format is meant to be both forwards- and
backwards-compatible.

\subsubsection{File section}

The file section consists of \prog{nfiles} file records. Each record
is \prog{frecsize} bytes long, and contains:

\vspace{1em}
\begin{tabular}{llrr}
Variable & Description                                       & Bytes & Type \\\hline
\prog{filename} & Name of file (possibly including full path)       & \prog{flen} & char *\\
\prog{format}   & Format code for file; see squid.h for definitions & 4    & \prog{sqd\_uint32} \\
\prog{flags}    & Optional behavior flags                           & 4    & \prog{sqd\_uint32} \\
\prog{bpl}      & Bytes per sequence data line                      & 4    & \prog{sqd\_uint32} \\
\prog{rpl}      & Residues per sequence data line                   & 4    & \prog{sqd\_uint32} \\\hline
\end{tabular}
\vspace{1em}

When a SSI file is written, \prog{frecsize} is equal to the sum of
the sizes above.  When a SSI file is read by a parser, it is possible
that \prog{frecsize} is larger than the parser expects, if the parser
is expecting an older version of the SSI format: additional fields
may be present, which increases \prog{frecsize}. The parser will only
try to understand the data up to the \prog{frecsize} it expected to
see, but still knows the absolutely correct \prog{frecsize} for
purposes of skipping around in the index file.

Normally the SSI index resides in the same directory as the sequence
data file(s), so \prog{filename} is relative to the location of the
SSI index. In the event this is not true, \prog{filename} can contain
a full path.

\prog{format} is a SQUID sequence file format code; e.g. something like 
\prog{SQFILE\_FASTA} or \prog{MSAFILE\_STOCKHOLM}. These constants are defined
in \prog{squid.h}.

Only one possible optional behavior flag is defined:

\vspace{1em}
\begin{tabular}{lll}
Flag             & Value& Note\\ \hline
\prog{SSI\_FAST\_SUBSEQ} & $1 \ll 0$ & Fast subseq retrieval is possible for this file.\\\hline
\end{tabular}
\vspace{1em}

When \prog{SSI\_FAST\_SUBSEQ} is set, \prog{bpl} and \prog{rpl} are
nonzero. They can be used to calculate the offset of subsequence
positions in the data file. This is described in the optional behavior
section below.

\subsubsection{Primary key section}

The primary key section consists of \prog{nprimary} records. Each
record is \prog{precsize} bytes long, and contains:

\vspace{1em}
\begin{tabular}{llrr}
Variable   & Description                                 & Bytes      & Type \\\hline
\prog{key}	   & Key name (seq name, identifier, accession) & \prog{plen}& char *\\
\prog{fnum}       & File number (0..nfiles-1)                   & 2          & \prog{sqd\_uint16}\\
\prog{offset1}    & Offset to start of record                   & \ddag      & \ddag \\
\prog{offset2}    & Offset to start of sequence data            & \ddag      & \ddag \\
\prog{len}        & Length of data (e.g. seq length, residues)  & 4          & \prog{sqd\_uint32} \\\hline
\end{tabular} 
\vspace{1em}

The offsets are sequence file offsets (indicated by \ddag). They are
4 bytes of type \prog{sqd\_uint32} normally, 8 bytes of type
\prog{sqd\_uint32} if \prog{SSI\_USE64} is set, and \prog{sizeof(fpos\_t)} 
bytes of type \prog{fpos\_t} if \prog{SSI\_FPOS\_T} is set.

\prog{offset2} and \prog{len} are only meaningful if \prog{SSI\_FAST\_SUBSEQ}
is set on this key's file. \prog{offset2} gives the absolute disk
position of line 0 in the sequence data. \prog{len} is necessary for
bounds checking in a subsequence retrieval, to be sure we don't try to
reposition the disk outside the valid data.

\subsubsection{Secondary key section}

The secondary key section consists of \prog{nsecondary} records. Each
record is \prog{srecsize} bytes long, and contains:

\vspace{1em}
\begin{tabular}{llrr}
Variable   & Description                                   & Bytes      & Type \\\hline
\prog{key}   & Key name (seq name, identifier, accession)  & \prog{slen}& char *\\
\prog{pkey}  & Primary key                                 &
\prog{plen}& char *\\\hline
\end{tabular}
\vspace{1em}

All data are kept with the primary key records. Secondary keys are
simply translated to primary keys, then the primary key has to be
looked up.

\subsection{Optional behaviors}

\subsubsection{Large sequence files and large indices: 64-bit operation}

Normally a SSI index file can be no larger than 2 GB, and can index
sequence files that are no larger than 2 GB each. This is due to
limitations in the ANSI C/POSIX standards, which were developed for
32-bit operating systems and filesystems. Most modern operating
systems allow larger 64-bit file sizes, but as far as I'm aware (Dec
2000), there are no standard interfaces yet for working with positions
(offsets) in large files. On many platforms, SSI can extend to full
64-bit capabilities, but on some platforms, it cannot. To understand
the limitations (of SSI, and possibly of my understanding) you need
to understand some details about what's happening behind the SSI API
and how I understand C API's to modern 64-bit OS's and hardware.

First, some information on ANSI C APIs for file positioning. ANSI C
provides the portable functions \prog{fseek()} and \prog{ftell()} for
manipulating simple offsets in a file. They store the offset in a
\prog{long} (which ranges up to 2 GB). The Standard says we're allowed
to do arithmetic on this value if the file is binary. ANSI C also
provides \prog{fgetpos()} and \prog{fsetpos()} which store file
positions in an opaque data type called \prog{fpos\_t}. Modern
operating systems with large file support define \prog{fpos\_t} in a
way that permits files $>$2 GB. However, \prog{fpos\_t} is an opaque
type. It has two disadvantages compared to a simple arithmetic type
like \prog{long}: first, we're not allowed to do arithmetic on it, and
second, we can't store it in a binary file in an
architecture-independent manner. We need both features for SSI,
unfortunately. \footnote{Surely the professional C community has the
same problem; does \emph{everyone} hack around \prog{fpos\_t}?}

Therefore we have to rely on system dependent features. Most operating
systems provide a non-compliant library call that returns an
arithmetic offset. Fully 64-bit systems typically give us a 64-bit
\prog{off\_t} and functions \prog{ftello()}/\prog{fseeko()} that work
with that offset.  Many systems provide a ``transitional interface''
where all normally named functions are 32-bits, but specially named
64-bit varieties are available: e.g. \prog{off\_t} is 32 bits, but
\prog{off64\_t} is 64 bits and we have functions \prog{ftello64()} and
\prog{fseeko64()}. Some systems provide a \prog{ftell64()} and
\prog{fseek64()} that work on offsets of type \prog{long long}. Many
popular systems may even provide more than one of these models,
depending on compiler flags. 

And, unfortunately, some systems provide none of these models (FreeBSD
for example). There, we will exploit the fact that most systems
(including FreeBSD) do in fact implement \prog{fpos\_t} as a simple
arithmetic type, such as an \prog{off\_t}, so we can misuse it.

At compile time, SQUID's \prog{./configure} script tests for the
system's capabilities for 64-bit file offsets, and configures a
section in the \prog{squidconf.h} file. (The configuration includes a
custom autoconf macro, \prog{SQ\_ARITHMETIC\_FPOS\_T()}, to test
\prog{fpos\_t} and define \prog{ARITHMETIC\_FPOS\_T} if it is.)  Four
possible 64-bit models are tested in the following order; if one of
them is possible, it will be used, and the constant
\prog{HAS\_64BIT\_FILE\_OFFSETS} is set.

\begin{enumerate}
\item has \prog{ftello()}, \prog{fseeko()}; sizeof(\prog{off\_t}) $= 8$.
\item has \prog{ftello64()}, \prog{fseeko64()}; sizeof(\prog{off64\_t}) $= 8$.
\item has \prog{ftell64()}, \prog{fseek64()}
\item \prog{fpos\_t} is an arithmetic 64-bit type; (mis)use
\prog{fgetpos()}, \prog{fsetpos()}.
\end{enumerate}



\subsubsection{Fast subsequence retrieval}

In some files (notably vertebrate chromosome contigs) the size of each
sequence is large. It may be slow to extract a subsequence by first
reading the whole sequence into memory -- or even prohibitive, if the
sequence is so large that it can't be stored in memory.

If the sequence data file is very consistently formatted so that each
line in each record (except the last one) is of the same length, in
both bytes and residues, we can determine a disk offset of the start
of any subsequence by direct calculation.
For example, a simple well-formatted FASTA
file with 50 residues per line would have 51 bytes per sequence line
(counting the '\verb+\0+') (\prog{bpl}=51, \prog{rpl}=50). Position $i$ in a sequence
$1..L$ will be on line $l = (i-1)/\mbox{\prog{rpl}}$, and line $l$ starts at
disk offset $l * \mbox{\prog{bpl}}$ relative to the start of the sequence
data. If there are no nonsequence characters in the data line except
the terminal '\verb+\0+' (which is true iff \prog{bpl} = \prog{rpl}+1 and 1 residue = 1
byte), position $i$ can be precisely found:

\[
\mbox{relative offset of residue $i$} =
\left((i-1)/\mbox{\prog{rpl}}\right)*\mbox{\prog{bpl}} + (i-1) \% \mbox{ \prog{rpl}}
\]

Even for sequence data lines with extra characters (e.g. spaces,
coordinates, whatever), fast subsequence retrieval is possible; a
parser can be positioned at the beginning of the appropriate line $l$,
which starts at residue $(l*\mbox{\prog{rpl}}) + 1$, and it can start reading
from there (e.g. the line that $i$ is on) rather than the beginning of
the whole sequence record.

The program that creates the index is responsible for determining if
\prog{bpl} and \prog{rpl} are consistent throughout a file; if so, it
may set the \prog{SSI\_FAST\_SUBSEQ} flag for the file. Then any record
whose primary key carries the optional data offset (\prog(offset2))
and sequence length data is available for subsequence position
calculations by \prog{SSIGetSubseqOffset()}. 

\end{document}